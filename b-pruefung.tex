% TeX für WBH B-Prüfungen
\documentclass[12pt,a4paper,bibliography=totocnumbered,listof=totocnumbered]{scrartcl}

% Support German annotation
\usepackage[ngerman]{babel}
\usepackage[utf8]{inputenc}

% Symbols:
% Pandoc imports the extensive `amsmath` collection of symbols
% for typesetting ordinary math.
\usepackage{amsmath}
% if you use exotic symbols you need to import specific packages, eg. for
% electrical engineering diagrams, musical notation, exotic currency symbols,
% the unspeakable rites of freemasonry etc.
\usepackage{amsfonts}
\usepackage{amssymb}


\usepackage{graphicx}
\usepackage{fancyhdr}
\usepackage{tabularx}
\usepackage{geometry}
\usepackage{setspace}
\usepackage[right]{eurosym}
\usepackage[printonlyused]{acronym}
\usepackage{subfig}
\usepackage{floatflt}
\usepackage[usenames,dvipsnames]{color}
\usepackage{colortbl}
\usepackage{paralist}
\usepackage{array}
\usepackage{titlesec}
\usepackage{parskip}
\usepackage[right]{eurosym}
%\usepackage{picins}
\usepackage[subfigure,titles]{tocloft}
\usepackage[pdfpagelabels=true]{hyperref}
\usepackage{helvet}
$if(highlighting-macros)$
$highlighting-macros$
$endif$
$if(verbatim-in-note)$
\usepackage{fancyvrb}
$endif$

\usepackage{listings}
\lstset{basicstyle=\footnotesize, captionpos=b, breaklines=true, showstringspaces=false, tabsize=2, frame=lines, numbers=left, numberstyle=\tiny, xleftmargin=2em, framexleftmargin=2em}
\makeatletter
\def\l@lstlisting#1#2{\@dottedtocline{1}{0em}{1em}{\hspace{1,5em} Lst. #1}{#2}}
\makeatother

\geometry{a4paper, top=27mm, left=20mm, right=40mm, bottom=35mm, headsep=10mm, footskip=12mm} % Vorgabe des 4cm Rand auf der rechten Seiten.

\hypersetup{unicode=false, pdftoolbar=true, pdfmenubar=true, pdffitwindow=false, pdfstartview={FitH},
	pdftitle={B-Prüfung},
	pdfauthor={$author$, Matrikelnummer: $matrikelnr$},
	pdfsubject={$studiengang$},
	pdfcreator={\LaTeX\ with package \flqq hyperref\frqq},
	pdfproducer={pdfTeX \the\pdftexversion.\pdftexrevision},
	pdfkeywords={B-Prüfung $aufgabencode$},
	pdfnewwindow=true,
	colorlinks=true,linkcolor=black,citecolor=black,filecolor=magenta,urlcolor=black}
\pdfinfo{/CreationDate (D:20170605133321)}
\renewcommand{\familydefault}{\sfdefault}

% Pandoc tightlisting
\providecommand{\tightlist}{%
  \setlength{\itemsep}{0pt}\setlength{\parskip}{0pt}}

\begin{document}

\titlespacing{\section}{0pt}{12pt plus 4pt minus 2pt}{-6pt plus 2pt minus 2pt}

% Kopf- und Fusszeile
\renewcommand{\sectionmark}[1]{\markright{#1}}
\renewcommand{\leftmark}{\rightmark}
\pagestyle{fancy}
\lhead{}
\chead{}
\rhead{\thesection\space\contentsname}
\lfoot{\tiny B-Prüfung des Studenten: $author$ (Matrikelnr.: $matrikelnr$) Studiengang: $studiengang$ - Prüfung: $aufgabencode$}
\cfoot{}
\rfoot{\ \linebreak Seite \thepage}
\renewcommand{\headrulewidth}{0.4pt}
\renewcommand{\footrulewidth}{0.4pt}

% Vorspann
\renewcommand{\thesection}{\Roman{section}}
\renewcommand{\theHsection}{\Roman{section}}
\pagenumbering{Roman}

% Pagebreak after each Section
\let\oldsection\section
\renewcommand\section{\clearpage\oldsection}

% ----------------------------------------------------------------------------------------------------------
% Titelseite
% ----------------------------------------------------------------------------------------------------------
\thispagestyle{empty}
\begin{center}
    $if(logo)$
    \includegraphics[scale=0.2]{$logo$}\\
	$endif$
    \vspace*{2cm}
	\Large
	\textbf{Studiengang:}\\
	\textbf{$studiengang$}\\
	\vspace*{2cm}
	\Huge
	\textbf{B-Aufgabe}\\
	\vspace*{0.5cm}
	\large
  $aufgabencode$ \\
	\vspace*{1cm}
	\textbf{$fach$}\\
	\vspace*{2cm}

	\vfill
	\normalsize
	\newcolumntype{x}[1]{>{\raggedleft\arraybackslash\hspace{0pt}}p{#1}}
	\begin{tabular}{x{6cm}p{7.5cm}}
		\rule{0mm}{5ex}\textbf{Student:} & $author$ \\
		\rule{0mm}{5ex}\textbf{Matrikelnummer:} & $matrikelnr$ \\
		\rule{0mm}{5ex}\textbf{Abgabedatum:} & $date$ \\
	\end{tabular}
\end{center}
\pagebreak

$if(abstract)$
\begin{abstract}
$abstract$
\end{abstract}
$endif$

$for(include-before)$
$include-before$

$endfor$

$if(toc)$
{
$if(colorlinks)$
\hypersetup{linkcolor=$if(toccolor)$$toccolor$$else$black$endif$}
$endif$
\setcounter{tocdepth}{$toc-depth$}
% ----------------------------------------------------------------------------------------------------------
% Verzeichnisse
% ----------------------------------------------------------------------------------------------------------
% TODO Typ vor Nummer
\renewcommand{\cfttabpresnum}{Tab. }
\renewcommand{\cftfigpresnum}{Abb. }
\settowidth{\cfttabnumwidth}{Abb. 10\quad}
\settowidth{\cftfignumwidth}{Abb. 10\quad}

\titlespacing{\section}{0pt}{12pt plus 4pt minus 2pt}{2pt plus 2pt minus 2pt}
\singlespacing
\rhead{INHALTSVERZEICHNIS}
\renewcommand{\contentsname}{II Inhaltsverzeichnis}
\phantomsection
\addcontentsline{toc}{section}{\texorpdfstring{II \hspace{0.35em}Inhaltsverzeichnis}{Inhaltsverzeichnis}}
\addtocounter{section}{1}
\tableofcontents
\pagebreak
}
$endif$
$if(lot)$
\rhead{VERZEICHNISSE}
\pagebreak
\listoftables
$endif$
$if(lof)$
% Workaround for the HEADING if you don't use a list of tables
$if(lot)$$else$\rhead{VERZEICHNISSE}$endif$
\listoffigures
\pagebreak
$endif$
$if(abk)$
% ----------------------------------------------------------------------------------------------------------
% Abkürzungen
% ----------------------------------------------------------------------------------------------------------
\section{Abkürzungsverzeichnis}
\begin{acronym}[OSGi] % längste Abkürzung steht in eckigen Klammern
	\setlength{\itemsep}{-\parsep} % geringerer Zeilenabstand
	\acro{OSGi}{Open Service Gateway initiative}
\end{acronym}
\newpage
$endif$

% ----------------------------------------------------------------------------------------------------------
% Inhalt
% ----------------------------------------------------------------------------------------------------------
% Abstände Überschrift
\titlespacing{\section}{0pt}{12pt plus 4pt minus 2pt}{-6pt plus 2pt minus 2pt}
\titlespacing{\subsection}{0pt}{12pt plus 4pt minus 2pt}{-6pt plus 2pt minus 2pt}
\titlespacing{\subsubsection}{0pt}{12pt plus 4pt minus 2pt}{-6pt plus 2pt minus 2pt}

% Kopfzeile
\renewcommand{\sectionmark}[1]{\markright{#1}}
\renewcommand{\subsectionmark}[1]{}
\renewcommand{\subsubsectionmark}[1]{}
\lhead{Lösung der Aufgabe: \thesection}
\rhead{} %hier kann die rechte Seite der Kopfzeile editiert werden!

\onehalfspacing
\renewcommand{\thesection}{\arabic{section}}
\renewcommand{\theHsection}{\arabic{section}}
\setcounter{section}{0}
\pagenumbering{arabic}
\setcounter{page}{1}


$body$



$if(biblatex)$
% ----------------------------------------------------------------------------------------------------------
% Literatur
% ----------------------------------------------------------------------------------------------------------
\renewcommand\refname{Quellenverzeichnis}
\bibliographystyle{myalpha}
\bibliography{bibo}
\pagebreak

\printbibliography$if(biblio-title)$[title=$biblio-title$]$endif$

$endif$

\newpage
\thispagestyle{empty}
\begin{center}
	\vspace*{5em}
	\huge\textbf{Erklärung}\\
\end{center}
\vspace{2em}
Hiermit versichere ich, dass ich meine Abschlussarbeit selbständig verfasst und keine anderen als die angegebenen Quellen und Hilfsmittel benutzt habe.

\vspace{4em}
\begin{minipage}{\linewidth}
	\begin{tabular}{p{15em}p{15em}}
		Datum: &  .......................................................\\
		& \centering ($author$)\\
	\end{tabular}
\end{minipage}

\end{document}
