% TeX für WBH B-Prüfungen
\documentclass[12pt,a4paper,bibliography=totocnumbered,listof=totocnumbered]{scrartcl}

% Support German annotation
$if(lang.de)$
\usepackage[ngerman]{babel}
\usepackage[utf8]{inputenc}
$endif$

% Symbols:
% Pandoc imports the extensive `amsmath` collection of symbols
% for typesetting ordinary math.
\usepackage{amsmath}
% if you use exotic symbols you need to import specific packages, eg. for
% electrical engineering diagrams, musical notation, exotic currency symbols,
% the unspeakable rites of freemasonry etc.
\usepackage{amsfonts}
\usepackage{amssymb}


\usepackage{graphicx}
\usepackage{fancyhdr}
\usepackage{tabularx}
\usepackage{geometry}
\usepackage{setspace}
\usepackage[right]{eurosym}
\usepackage[printonlyused]{acronym}
\usepackage{subfig}
\usepackage{floatflt}
\usepackage[usenames,dvipsnames]{color}
\usepackage{colortbl}
\usepackage{paralist}
\usepackage{array}
\usepackage{titlesec}
\usepackage{parskip}
\usepackage[right]{eurosym}
%\usepackage{picins}
\usepackage[subfigure,titles]{tocloft}
\usepackage[pdfpagelabels=true]{hyperref}
\usepackage{helvet}
% This two Packages are needet for Pandoc Table support. Issue is opened: https://github.com/jgm/pandoc/issues/1023
\usepackage{longtable}
\usepackage{booktabs}
% For using SI units https://www.ctan.org/pkg/siunitx
\usepackage{siunitx}

% blockquote
\definecolor{blockquote-border}{RGB}{221,221,221}
\definecolor{blockquote-text}{RGB}{119,119,119}
\usepackage{mdframed}
\newmdenv[rightline=false,bottomline=false,topline=false,linewidth=3pt,linecolor=blockquote-border,skipabove=\parskip]{customblockquote}
\renewenvironment{quote}{\begin{customblockquote}\list{}{\rightmargin=0em\leftmargin=0em}%
\item\relax\color{blockquote-text}\ignorespaces}{\unskip\unskip\endlist\end{customblockquote}}

% Syntax Highligting with colors
$if(highlighting-macros)$
$highlighting-macros$
$endif$
$if(verbatim-in-note)$
\usepackage{fancyvrb}
$endif$


\usepackage{listings}
\lstset{basicstyle=\footnotesize, captionpos=b, breaklines=true, showstringspaces=false, tabsize=2, frame=lines, numbers=left, numberstyle=\tiny, xleftmargin=2em, framexleftmargin=2em}
\makeatletter
\def\l@lstlisting#1#2{\@dottedtocline{1}{0em}{1em}{\hspace{1,5em} Lst. #1}{#2}}
\makeatother

\geometry{a4paper, top=27mm, left=20mm, right=40mm, bottom=35mm, headsep=10mm, footskip=12mm} % Vorgabe des 4cm Rand auf der rechten Seiten.

\hypersetup{unicode=false, pdftoolbar=true, pdfmenubar=true, pdffitwindow=false, pdfstartview={FitH},
	pdftitle={$if(title)$$title$:$endif$ $if(aufgabe.code)$$aufgabe.code$ - $endif$$if(author.name)$$author.name$$endif$},
$if(author)$
	pdfauthor={$if(author.name)$$author.name$$else$$author$$endif$$if(author.matrikelnr)$, Matrikelnummer: $author.matrikelnr$$endif$},
$endif$
$if(studium.studiengang)$
	pdfsubject={Studiengang: $studium.studiengang$},
$endif$
	pdfcreator={\LaTeX\ with package \flqq hyperref\frqq},
	pdfproducer={pdfTeX \the\pdftexversion.\pdftexrevision},
	pdfkeywords={$if(aufgabe.typ)$$aufgabe.typ$$endif$, $if(author.matrikelnr)$$author.matrikelnr$$endif$, $if(aufgabe.code)$$aufgabe.code$$endif$ $if(keywords)$ $for(keywords)$$keywords$$sep$, $endfor$ $endif$},
	pdfnewwindow=true,
	pdflang=$if(lang)$$lang$$else$de$endif$,
	pdfdisplaydoctitle=true, colorlinks=true, linkcolor=black, citecolor=gray, filecolor=magenta,	urlcolor=black}
% \pdfinfo{/CreationDate (D:20170605133321)}
\renewcommand{\familydefault}{\sfdefault}

% Pandoc tightlisting
\providecommand{\tightlist}{%
  \setlength{\itemsep}{0pt}\setlength{\parskip}{0pt}}

\begin{document}

\titlespacing{\section}{0pt}{12pt plus 4pt minus 2pt}{-6pt plus 2pt minus 2pt}

% ----------------------------------------------------------------------------------------------------------
% Kopf und Fußzeile
% ----------------------------------------------------------------------------------------------------------
\renewcommand{\sectionmark}[1]{\markright{#1}}
\renewcommand{\leftmark}{\rightmark}
\pagestyle{fancy}
\lhead{}
\chead{}
\rhead{\thesection\space\contentsname}
\lfoot{\tiny $if(aufgabe.typ)$$aufgabe.typ$ des Studenten: $endif$$if(author.name)$$author.name$$endif$ $if(author.matrikelnr)$(Matrikelnr.: $author.matrikelnr$)$endif$ $if(studium.studiengang)$Studiengang: $studium.studiengang$$endif$ $if(aufgabe.code)$- Prüfung: $aufgabe.code$ $endif$}
\cfoot{}
\rfoot{\ \linebreak Seite \thepage}
\renewcommand{\headrulewidth}{0.4pt}
\renewcommand{\footrulewidth}{0.4pt}

% Vorspann
\renewcommand{\thesection}{\Roman{section}}
\renewcommand{\theHsection}{\Roman{section}}
\pagenumbering{Roman}

% Pagebreak after each Section
\let\oldsection\section
\renewcommand\section{\clearpage\oldsection}

% ----------------------------------------------------------------------------------------------------------
% Titelseite
% ----------------------------------------------------------------------------------------------------------
\thispagestyle{empty}
\begin{center}
  $if(logo)$
    \includegraphics[scale=0.2]{$logo$}\\
	$endif$
    \vspace*{2cm}
	\Large
	$if(studium.studiengang)$
		\textbf{Studiengang:}\\
		\textbf{$studium.studiengang$}\\
		\vspace*{2cm}
	$endif$
	\Huge
	$if(aufgabe.typ)$
		\textbf{$aufageb.typ$}\\
		\vspace*{0.5cm}
	$endif$
	$if(title)$
		\textbf{$title$} \\
	$endif$
	\vspace*{0.3cm}
	\large
  $if(aufgabe.code)$
		$aufgabe.code$ \\
	$endif$
	\vspace*{1cm}
	$if(studium.fach)$
		\textbf{$studium.fach$}\\
	$endif$
	\vspace*{2cm}

	\vfill
	\normalsize
	\newcolumntype{x}[1]{>{\raggedleft\arraybackslash\hspace{0pt}}p{#1}}
	\begin{tabular}{x{6cm}p{7.5cm}}
		$if(author.name)$
			\rule{0mm}{5ex}\textbf{Student:} & $author.name$
			$if(author.email)$
				\newline $author.email$
			$endif$ \\
		$endif$
		$if(author.matrikelnr)$
			\rule{0mm}{5ex}\textbf{Matrikelnummer:} & $author.matrikelnr$ \\
		$endif$
		$if(date)$
			\rule{0mm}{5ex}\textbf{Abgabedatum:} & $date$ \\
		$else$
			\rule{0mm}{5ex}\textbf{Abgabedatum:} & \today \\
		$endif$
	\end{tabular}
\end{center}
\pagebreak

$if(abstract)$
	% ----------------------------------------------------------------------------------------------------------
	% Abstract
	% ----------------------------------------------------------------------------------------------------------
	\begin{abstract}
		$abstract$
	\end{abstract}
$endif$

% ----------------------------------------------------------------------------------------------------------
% Content
% ----------------------------------------------------------------------------------------------------------
$for(include-before)$
	$include-before$
$endfor$

$if(toc)$
	{
	$if(colorlinks)$
		\hypersetup{linkcolor=$if(toccolor)$$toccolor$$else$black$endif$}
	$endif$

	\setcounter{tocdepth}{$toc-depth$}

	% ----------------------------------------------------------------------------------------------------------
	% Verzeichnisse
	% ----------------------------------------------------------------------------------------------------------
	% TODO Typ vor Nummer
	\renewcommand{\cfttabpresnum}{Tab. }
	\renewcommand{\cftfigpresnum}{Abb. }
	\settowidth{\cfttabnumwidth}{Abb. 10\quad}
	\settowidth{\cftfignumwidth}{Abb. 10\quad}

	\titlespacing{\section}{0pt}{12pt plus 4pt minus 2pt}{2pt plus 2pt minus 2pt}
	\singlespacing
	\rhead{$if(lang.de)$INHALTSVERZEICHNIS$else$TABLE OF CONTENTS$endif$}
	\renewcommand{\contentsname}{I $if(lang.de)$Inhaltsverzeichnis$else$Table of contents$endif$}
	\phantomsection
	\addcontentsline{toc}{section}{\texorpdfstring{I \hspace{0.35em}$if(lang.de)$Inhaltsverzeichnis$else$Table of contents$endif$}{$if(lang.de)$Inhaltsverzeichnis$else$Table of contents$endif$}}
	\addtocounter{section}{1}
	\tableofcontents
	\pagebreak
	}
$endif$

$if(lot)$
	\rhead{$if(lang.de)$TABELLENVERZEICHNIS$else$LIST OF TABLES$endif$}
	\pagebreak
	\listoftables
$endif$
$if(lof)$
	$if(lot)$
		% Workaround for the HEADING if you don't use a list of tables
	$else$
		\rhead{$if(lang.de)$ABBILDUNGSVERZEICHNIS$else$LIST OF FIGURES$endif$}
	$endif$
	\listoffigures
	\pagebreak
$endif$

$if(abk)$
	% ----------------------------------------------------------------------------------------------------------
	% Abkürzungen
	% ----------------------------------------------------------------------------------------------------------
	\section{Abkürzungsverzeichnis}
	\begin{acronym}[OSGi] % längste Abkürzung steht in eckigen Klammern
		\setlength{\itemsep}{-\parsep} % geringerer Zeilenabstand
		\acro{OSGi}{Open Service Gateway initiative}
	\end{acronym}
	\newpage
$endif$

% ----------------------------------------------------------------------------------------------------------
% Inhalt
% ----------------------------------------------------------------------------------------------------------
% Abstände Überschrift
\titlespacing{\section}{0pt}{12pt plus 4pt minus 2pt}{4pt plus 2pt minus 2pt}
\titlespacing{\subsection}{0pt}{12pt plus 4pt minus 2pt}{2pt plus 2pt minus 2pt}
\titlespacing{\subsubsection}{0pt}{12pt plus 4pt minus 2pt}{2pt plus 2pt minus 2pt}

% Kopfzeile
\renewcommand{\sectionmark}[1]{\markright{#1}}
\renewcommand{\subsectionmark}[1]{}
\renewcommand{\subsubsectionmark}[1]{}
\lhead{$if(lang.de)$Abschnitt$else$Chapter$endif$ \thesection}
\rhead{} %hier kann die rechte Seite der Kopfzeile editiert werden!

\onehalfspacing
\renewcommand{\thesection}{\arabic{section}}
\renewcommand{\theHsection}{\arabic{section}}
\setcounter{section}{0}
\pagenumbering{arabic}
\setcounter{page}{1}
%\renewcommand{\includegraphics}[1][]{\includegraphics[width=0.9\columnwidth,keepaspectratio]{#1}}

$body$


$if(biblatex)$
	% ----------------------------------------------------------------------------------------------------------
	% Literatur
	% ----------------------------------------------------------------------------------------------------------
	\pagebreak
	\lhead{}
	\rhead{QUELLENVERZEICHNIS} %hier kann die rechte Seite der Kopfzeile editiert werden!
	\renewcommand{\refname}{Quellenverzeichnis}	

	\ bibliographystyle{myalpha}
	\bibliography{bibo}
	\pagebreak

	%\printbibliography$if(biblio-title)$[title=$biblio-title$]$endif$
$endif$

$if(insurance)$
	% ----------------------------------------------------------------------------------------------------------
	% Erklärung zur selbstständigkeit
	% ----------------------------------------------------------------------------------------------------------
	\newpage
	\thispagestyle{empty}
	\begin{center}
		\vspace*{5em}
		\huge\textbf{Erklärung}\\
	\end{center}
	\vspace{2em}
	Hiermit versichere ich, dass ich meine Abschlussarbeit selbständig verfasst und keine anderen als die angegebenen Quellen und Hilfsmittel benutzt habe.

	\vspace{4em}
	\begin{minipage}{\linewidth}
		\begin{tabular}{p{15em}p{15em}}
			Datum: &  .......................................................\\
			& \centering ($author.name$)\\
		\end{tabular}
	\end{minipage}
$endif$

\end{document}
